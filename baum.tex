\section{Einführung}
    \begin{itemize}
        \item[Bisher:] Wörter, d.h. gerichtete, zusammenhängende Grad1-Graphen\\
        oder: Funktion $\{1,\dots,n\}\rightarrow\Sigma$
        \item[jetzt:] Bäume
    \end{itemize}
    \subsection{Bäume}
        Was ist ein Baum?\\
        Graph mit folgenden Eigenschaften:
        \begin{itemize}
            \item azyklisch
            \item zusammenhängend (sonst ``Wald'')
            \item gerichtet
            \item gelabelte Knoten
            \item Rang (1 - Wörter, 2 - Binärbäume,\dots)\\
                je nach Definition:
            \begin{itemize}
                \item label-abhängig
                \item ohne Rang
            \end{itemize}
            \item endlich oder unendlich
        \end{itemize}
        Wir betrachten: endliche Binärbäume
    \subsection{Baumsprachen}
        Baumsprache $L\subseteq T^*_\Sigma$
\section{Knotenadressierung}
    \begin{itemize}
        \item bei Wörtern: $\{0,\dots,n\}\rightarrow\Sigma$
        \item bei Bäumen: $A\rightarrow\Sigma$
    \end{itemize}
    wobei $A\subseteq\{0,1\}^*,\ A$ abgeschlossen unter Präfixbildung und $w0\in A\Leftrightarrow w1\in A$
    \subsection{Beispiel}
        \begin{tikzpicture}[node distance=1.5cm]
            \node[state] (e) {$\epsilon$};
            \node[state] (0) [below left of=e] {0};
            \node[state] (00) [below left of=0] {00};
            \node[state] (01) [below right of=0] {01};
            \node[state] (010) [below left of=01] {010};
            \node[state] (011) [below right of=01] {011};
            \node[state] (1) [below right of=e] {1};

            \draw (e) -- (0);
            \draw (e) -- (1);
            \draw (0) -- (00);
            \draw (0) -- (01);
            \draw (01) -- (010);
            \draw (01) -- (011);
        \end{tikzpicture}
        \vspace*{-3cm}\\\hspace*{7cm}z.B. $t(01)=a$\vspace*{3cm}
\section{Konkatenation}
    im Gegensatz zu Wörtern nicht eindeutig:\vspace{-1cm}\\
    \begin{tikzpicture}[node distance=1.5cm]
        \node[state] (a) {};
        \node[state] (b) [below left of=a] {};
        \node[state] (c) [below right of=a] {};
        \node (cd) [above   right of=c]{$\cdot$};
        \draw (a) -- (b);
        \draw (a) -- (c);
    \end{tikzpicture}
    \begin{tikzpicture}[node distance=1.5cm]
        \node[state] (a) {};
        \node[state] (b) [below left of=a] {};
        \node[state] (c) [below right of=a] {};
        \node (cd) [above   right of=c]{$=$};
        \draw (a) -- (b);
        \draw (a) -- (c);
    \end{tikzpicture}
    \begin{tikzpicture}[node distance=1.5cm]
        \node[state] (a) {};
        \node[state] (b) [below left of=a] {};
        \node[state] (c) [below right of=a] {};
        \draw (a) -- (b);
        \draw (a) -- (c);
        \node[state] (a1) [below left of=b] {};
        \node[state] (b1) [below left of=a1] {};
        \node[state] (c1) [below right of=a1] {};
        \draw (a1) -- (b1);
        \draw (a1) -- (c1);

        \draw (b) -- (a1);
        \node (cd) [below right of=c]{$\leftarrow$ kein Binärbaum};
    \end{tikzpicture}
    \subsection{Kontexte}
        Idee: genau ein Blatt ist Loch ($\rightarrow$ ``Kontext'')\\
        \begin{tikzpicture}[node distance=1.5cm]
        \node[state] (a) {};
        \node[draw,rectangle,minimum size=0.5cm] (b) [below left of=a] {};
        \node[state] (c) [below right of=a] {};
        \node (cd) [above   right of=c]{$\cdot$};
        \draw (a) -- (b);
        \draw (a) -- (c);
    \end{tikzpicture}
    \begin{tikzpicture}[node distance=1.5cm]
        \node[state] (a) {};
        \node[state] (b) [below left of=a] {};
        \node[state] (c) [below right of=a] {};
        \node (cd) [above   right of=c]{$=$};
        \draw (a) -- (b);
        \draw (a) -- (c);
    \end{tikzpicture}
    \begin{tikzpicture}[node distance=1.5cm]
        \node[state] (a) {};
        \node[state] (c) [below right of=a] {};
        \draw (a) -- (b);
        \draw (a) -- (c);
        \node[state] (a1) [below left of=a] {};
        \node[state] (b1) [below left of=a1] {};
        \node[state] (c1) [below right of=a1] {};
        \draw (a1) -- (b1);
        \draw (a1) -- (c1);

        \draw (a) -- (a1);
    \end{tikzpicture}
