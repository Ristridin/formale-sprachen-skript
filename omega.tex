\section{Einführung}
    \subsection{Sprachen bisher}
        Teilmenge von $\Sigmas=\Sigma^0\cup\Sigma^1\cup\dots$\\
        Ein Wort aus $\Sigma^n$ kann als Funktion $\{1,\dots,n\}\rightarrow\Sigma$ aufgefasst werden.\\
    \subsection{unendliche Sprachen}
        Verallgemeinerung auf $\mathds{N}\rightarrow\Sigma$\\
        Menge unendlicher Wörter über $\Sigma$: $\Sigma^\omega$\\
        $\Sigma^\infty:=\Sigma^\omega\cup\Sigmas$
        \subsubsection{Beispiel}
            $\{ab\}^*\{bba\}^\omega\cup\{aa\}^\omega$
            \begin{itemize}
                \item Es gibt kein letztes Zeichen
                \item Konkatenation ist eingeschränkt
            \end{itemize}
\section{Definition: $\omega$-reguläre Sprachen}
    \begin{itemize}
        \item $A^\omega$ ist $\omega$-regulär, wenn $A\not=\emptyset$ und regulär
        \item $AB$ ist $\omega$-regulär, wenn $A$ regulär und $B$ $\omega$-regulär
        \item $A\cup B$ ist $\omega$-regulär, wenn $A$ und $B$ $\omega$-regulär
    \end{itemize}
    \subsection{Beispiel: Komplementbildung}
        $\left(\{0,1\}^*\{0\}^\omega\right)^c=\{0,1\}^*\{1\}^\omega\cup\left(\{0,1\}^*\setminus0^*\right)^\omega$
\section{$\omega$-Automaten}
    \subsection{Problem}
        Für endliche Wörter akzeptiert ein Automat, wenn er \emph{nach} dem letzten Zeichen in akzeptierendem Zustand ist. Das geht im $\omega$-Fall nicht (kein letztes Zeichen).
    \subsection{Definition}
        $\mathcal{A}=\left(Q,\Sigma,q_0,\delta,Acc\right)$
        \begin{itemize}
            \item $Q$ Zustände
            \item $\Sigma$ Alphabet
            \item $q_0\in Q$ Startzustand
            \item $\delta\subseteq Q\times\Sigma\times Q$ (nicht deterministisch)\\$\delta: Q\times\Sigma\rightarrow Q$ (deterministisch)
            \item $Acc$ Akzeptanzbedingung, je nach Automatentyp
        \end{itemize}
    \subsection{Definition}
        Sei $x\in\Sigmao$, dann ist $\rho\in Q^\omega$ ein Lauf auf $x\Leftrightarrow (\rho_i,x_i,\rho_{i+1})\in\delta$ bzw. $\rho_{i+1}=\delta(\rho_i,x_i)$\\
        $In(\rho):=\{q\in Q\mid \exists^\omega i: \rho_i=q\}$ ``es existieren unendlich viele''
    \subsection{Akzeptanzbedingungen}
        \subsubsection{Büchi}
            $Acc=F\subseteq Q$\\
            $x\in L(\mathcal{A})\Leftrightarrow In(\rho)\cap F\not=\emptyset$\\
            \vspace*{-2cm}\\\hspace*{7cm}
                \begin{tikzpicture}[->,node distance=2cm]
                    \node[initial,state] (q0) {$q_0$};
                    \node[accepting, state] [right of=q0] (q1) {$q_1$};

                    \draw (q0) edge[loop above] node[above]{0,1} (q0);
                    \draw (q0) edge node[above]{0} (q1);
                    \draw (q1) edge[loop above] node[above]{0} (q1);
                \end{tikzpicture}
        \subsubsection{Muller}
            $Acc=\mathcal{F}\subseteq 2^Q$\\
            $x\in L(\mathcal{A})\Leftrightarrow \bigvee\limits_{F\in\mathcal{F}} In(\rho)=F$
            \\\vspace*{-2cm}\\\hspace*{5cm}
                \begin{tikzpicture}[->,node distance=2cm]
                    \node[initial,state] (q0) {$q_0$};
                    \node[state] [right of=q0] (q1) {$q_1$};

                    \draw (q0) edge[loop above] node[above]{0} (q0);
                    \draw (q0) edge[bend left] node[above]{1} (q1);
                    \draw (q1) edge[loop above] node[above]{1} (q1);
                    \draw (q1) edge[bend left] node[below]{0} (q0);
                \end{tikzpicture}
            \\\vspace*{-2cm}\\\hspace*{10cm} $\mathcal{F}=\{\{q_0\},\{q_0,q_1\}\}$\vspace{1cm}
        \subsubsection{Rabin}
            $Acc=\Omega\in(2^Q\times 2^Q)^*$\\
            $\Omega=\{(E_1,F_1),(E_2,F_2),\dots,(E_n,F_n)\}$\\
            $x\in L(\mathcal{A})\Leftrightarrow\bigvee\limits_{i=1}^n\left(In(\rho)\cap E_i=\emptyset\wedge In(\rho)\cap F_i\not=\emptyset\right)$
        \subsubsection{Streett}
            $Acc$ wie bei Rabin, aber:\\
            $x\in L(\mathcal{A})\Leftrightarrow\bigwedge\limits_{i=1}^n\left(In(\rho)\cap E_i\not=\emptyset\vee In(\rho)\cap F_i=\emptyset\right)$
    \subsection{Satz}
        Folgende Modelle sind gleichmächtig: Muller,Rabin,Streett, nicht-deterministischer Büchi\\
        Deterministischer Büchi ist echt schwächer. Trennbeispiel:\\
        \begin{tikzpicture}[->,node distance=2cm]
            \node[initial,state] (q0) {$q_0$};
            \node[accepting, state] [right of=q0] (q1) {$q_1$};

            \draw (q0) edge[loop above] node[above]{0,1} (q0);
            \draw (q0) edge node[above]{0} (q1);
            \draw (q1) edge[loop above] node[above]{0} (q1);
        \end{tikzpicture} \hspace{1cm}nicht determinisierbar
    \subsection{Satz}
        nicht-deterministischer Büchi $\approx\omega$-reguläre Sprachen
    \subsection{Komplementabschluss}
        Ist nicht-deterministischer Büchi unter Komplement abgeschlossen?\\
        Für deterministische Automaten klar:
        \begin{itemize}
            \item Büchi $F\rightarrow Q\setminus F?$ Nein!
            \item Muller $\mathcal{F}\rightarrow 2^Q\setminus\mathcal{F} \checkmark$
        \end{itemize}
    \subsection{Satz}
        Es existiert eine Umwandlung vom nicht-deterministischen Büchi zum deterministischen Muller-Automaten.
    \subsection{Satz}
        $\omega$-reguläre Sprachen sind unter Komplement abgeschlossen (klar, s.o.)
\section{Logik für $\omega$-reguläre Sprachen}
    $MSO$ (und $FO$) Formeln können auf $\omega$-Wörtern interpretiert werden.\\
    Zum Beispiel: $\ \forall x\exists z (x<z\wedge Q_az)\ $ ``unendlich viele a's''
    \subsection{Satz}
        $MSO=\omega$-regulär
\section{LTL - Linear Time Logic}
    \begin{itemize}
        \item oft kompakter als $MSO$ bzw. $FO$
        \item wird in Praxis verwendet
    \end{itemize}
    \subsection{Definition}
        \subsubsection{LTL-Syntax}
            \begin{itemize}
                \item $a\in\Sigma$ sind atomare Formeln
                \item sind $\varphi$ und $\psi$ LTL-Formeln, so auch $\neg\varphi,\ \varphi\wedge\psi,\ X\varphi,\ G\varphi,\ F\varphi,\ \psi U\varphi$
                \item auch auftretende Schreibweise:\\
                    \begin{itemize}
                        \item $X\varphi\rightarrow\circ\varphi$
                        \item $G\varphi\rightarrow\square\varphi$
                        \item $F\varphi\rightarrow\diamond\varphi$
                    \end{itemize}
            \end{itemize}
        \subsubsection{LTL-Semantik}
            \begin{itemize}
                \item $x\models a:\Leftrightarrow x_1=a$
                \item $x\models \neg\varphi:\Leftrightarrow x\not\models\varphi$
                \item $x\models \varphi\wedge\psi:\Leftrightarrow x\models\varphi\wedge x\models\psi$
                \item $x\models X\varphi:\Leftrightarrow x_2x_3\dots\models\varphi$
                \item $x\models G\varphi:\Leftrightarrow x\models\neg F\neg\varphi$
                \item $x\models F\varphi:\Leftrightarrow \exists i:x_ix_{i+1}\dots\models\varphi$
                \item $x\models \varphi U\psi:\Leftrightarrow\exists i\forall k<i: x_ix_{i+1}\dots\models\psi\wedge x_kx_{k+1}\dots\models\varphi$ ``until''
            \end{itemize}
    \subsection{Satz}
        LTL$=FO[<]$\pagebreak
\section{Anwendungsbeispiel:: Model-Checking}
    \subsection{Idee}
        \begin{itemize}
            \item Erfüllt ein System eine Eigenschaft?
            \item System: endlich, nicht-terminierend
            \item Kripke-Struktur $\rightarrow$ Büchi-Automat
            \item Eigenschaft: z.B. LTL-Formel $\varphi$
            \item Aufgabe: gilt $L(\mathcal{A})\subseteq L(\varphi)$\\
            $\Leftrightarrow L(\mathcal{A})\cap\widehat{L(\varphi)}=\emptyset$\\
            $\Leftrightarrow \underbrace{L(\mathcal{A})\cap L(\neg\varphi)}_{\text{Büchi-Automat}}=\emptyset$
            \item $\rightarrow$ Leerheitstest für Büchi-Automaten:
            \begin{enumerate}
                \item Finde von Startzustand erreichbaren Endzustand ($\rightarrow$ Graphenerreichbarkeit)
                \item gibt es Endzustandsschleife?
            \end{enumerate}
        \end{itemize}
