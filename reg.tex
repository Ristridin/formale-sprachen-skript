    \subsection{Darstellungen}
        DEA, NEA, reg. Ausdrücke, endl. Monoide, Typ-3-Grammatik, Logik
\section{Definition: Transformationsmonoid}
        Sei $M$ ein DEA, $Q$ Zustände, $w\in\Sigmas$\\
        $f_w:Q\rightarrow Q: q\mapsto \delta^*(q,w)$\\
        $Trans(M):=\{f_w\mid w\in\Sigmas\}$
    \subsection{Lemma}
        $Trans(M)$ ist endl. Monoid mit:
        \begin{itemize}
            \item $f_\epsilon=id_Q$ neutrales Element
            \item $f_x\cdot f_y=f_{xy}$
            \item $|Trans(M)|\leq |Q|^{|Q|}$
        \end{itemize}
\section{Satz}
    $L$ reg. $\Leftrightarrow L$ wird von endlichem Monoid $H$ erkannt
    \subsection{Beweis}
        \begin{itemize}
            \item[$\Rightarrow$:] $Trans(M)$ erkennt $L$ mit $\varphi:\Sigmas\rightarrow Trans(M):w\mapsto f_w,\\ X=\{f:Q\rightarrow Q\mid f(q_0)\in F\}$\\
            $w\in L\Rightarrow\exists$ Lauf von $q_0$ nach $f\in F\Leftrightarrow f_w(q_0)\in F\Leftrightarrow\varphi(w)\in X$
            \item[$\Leftarrow$:] Konstruiere $M$ aus $H$:\\
            $M=(Q,\Sigma,\delta,q_0,F),Q=H,q_0=1,F=X,\delta(q,a)=q\cdot\varphi(a)$\qed
        \end{itemize}
\section{Satz}
    DFA $M$ minimal $\Leftrightarrow Synt(L(M))\simeq Trans(M)$
    \subsection{Beweis}
        \begin{itemize}
            \item[$\Leftarrow$:] ergibt sich aus Satz oben ($Trans(M)$ kann nicht kleiner sein als das $Synt(L(M))$)
            \item[$\Rightarrow$:] z.z. $M$ min $\Rightarrow (f_x=f_y\Leftrightarrow x\sim_L y)$
            \begin{itemize}
                \item[$\Rightarrow$:] Folgt aus Satz
                \item[$\Leftarrow$:] ang. $x\sim_L y$ aber $f_x\not=f_y$\\
                also $\exists u,v: q_1=f_{uxv}(q_0)\not=f_{uyv}(q_0)=q_2$\\
                also $\forall v': f_{v'}(q_1)\in F\Leftrightarrow f_{v'}(q_2)\in F$\\
                dann aber Automat nicht minimal $\lightning$ \qed
            \end{itemize}
            $\Rightarrow$ Syntaktisches Monoid aus Minimalautomat berechenbar.
        \end{itemize}
