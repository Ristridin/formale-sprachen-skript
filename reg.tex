\section{Grundlagen}
    \subsection{Darstellungen}
        DEA, NEA, reg. Ausdrücke, endl. Monoide, Typ-3-Grammatik, Logik
    \subsection{Definition: Transformationsmonoid}
        Sei $M$ ein DEA, $Q$ Zustände, $w\in\Sigmas$\\
        $f_w:Q\rightarrow Q: q\mapsto \delta^*(q,w)$\\
        $Trans(M):=\{f_w\mid w\in\Sigmas\}$
    \subsection{Lemma}
        $Trans(M)$ ist endl. Monoid mit:
        \begin{itemize}
            \item $f_\epsilon=id_Q$ neutrales Element
            \item $f_x\cdot f_y=f_{xy}$
            \item $|Trans(M)|\leq |Q|^{|Q|}$
        \end{itemize}
    \subsection{Satz}
        $L$ reg. $\Leftrightarrow L$ wird von endlichem Monoid $H$ erkannt
        \subsubsection{Beweis}
            \begin{itemize}
                \item[$\Rightarrow$:] $Trans(M)$ erkennt $L$ mit $\varphi:\Sigmas\rightarrow Trans(M):w\mapsto f_w,\\ X=\{f:Q\rightarrow Q\mid f(q_0)\in F\}$\\
                $w\in L\Rightarrow\exists$ Lauf von $q_0$ nach $f\in F\Leftrightarrow f_w(q_0)\in F\Leftrightarrow\varphi(w)\in X$
                \item[$\Leftarrow$:] Konstruiere $M$ aus $H$:\\
                $M=(Q,\Sigma,\delta,q_0,F),Q=H,q_0=1,F=X,\delta(q,a)=q\cdot\varphi(a)$\qed
            \end{itemize}
    \subsection{Satz}
        DFA $M$ minimal $\Leftrightarrow Synt(L(M))\simeq Trans(M)$
        \subsubsection{Beweis}
            \begin{itemize}
                \item[$\Leftarrow$:] ergibt sich aus Satz oben ($Trans(M)$ kann nicht kleiner sein als das $Synt(L(M))$)
                \item[$\Rightarrow$:] z.z. $M$ min $\Rightarrow (f_x=f_y\Leftrightarrow x\sim_L y)$
                \begin{itemize}
                    \item[$\Rightarrow$:] Folgt aus Satz
                    \item[$\Leftarrow$:] ang. $x\sim_L y$ aber $f_x\not=f_y$\\
                    also $\exists u,v: q_1=f_{uxv}(q_0)\not=f_{uyv}(q_0)=q_2$\\
                    also $\forall v': f_{v'}(q_1)\in F\Leftrightarrow f_{v'}(q_2)\in F$\\
                    dann aber Automat nicht minimal $\lightning$ \qed
                \end{itemize}
                $\Rightarrow$ Syntaktisches Monoid aus Minimalautomat berechenbar.
            \end{itemize}
\section{Logik}
    Wir betrachten logische Formeln, die auf Wörtern operieren.\\
    Sei $w\in\Sigmas$ und $\varphi$ logische Formel:\\
    $w\models \varphi$, wenn $w$ Modell für $\varphi$, also $\varphi$ wahr unter $w$.
    \subsection{Beispiel}
        \begin{itemize}
            \item $\varphi_1=\exists x\exists y( \underbrace{\forall z (z\geq x)}_{x=1}\wedge Q_ax\wedge \underbrace{\forall z(z\leq y)}_{y=|w|}\wedge Q_by),\ x,y,z\in\{1,\dots,|w|\}$
            \begin{itemize}
                \item $L(\varphi_1)=a\Sigmas b$
                \item Quantifizierung erster Ordnung
                \item Variablen: Positionen im Wort
                \item Prädikate: $<,\ Q_a,\ Q_b$; $Q_ax$ wahr $\Leftrightarrow a$ an Stelle $x$
                \item alle Variablen müssen gebunden sein
            \end{itemize}
            \item $\varphi_2=\exists X\left(\forall z(z\geq x)\Rightarrow X(x)\right)\wedge \forall x\left(\forall z(z\leq x)\Rightarrow \neg X(x)\right)\wedge\\\hspace*{1.2cm} \forall x\forall y\left(\left((x<y)\wedge\forall z (z>x\Rightarrow z\geq y)\right)\Rightarrow\left(X(x)\Leftrightarrow \neg X(y)\right)\right)$
            \begin{itemize}
                \item $L(\varphi_2)=\left(\Sigma^2\right)^*$
                \item $\exists X$ ist Quantor zweiter Ordnung
                \item $X$ ist einstelliges Prädikat, d.h. monadisch
                \item $X$ beschreibt Menge von Positionen im Wort
                \item erste Zeile: erste Position im Wort in $X$, letzte nicht
                \item zweite Zeile: benachbarte Position $\Rightarrow$ genau eine in $X$
            \end{itemize}
        \end{itemize}
        \subsubsection{Anmerkungen}
            \begin{itemize}
                \item $x=y\equiv x\le y\wedge y\le x$
                \item $x<y\equiv x\le y\wedge \neg(x=y)$
                \item $x+1=y\equiv x<y\wedge\forall z(z>x\Rightarrow z\geq y)$
                \item aber $\leq$ nicht durch $+1$ simulierbar in $FO$ (aber $MSO[+1]=MSO[<]$)
                \item auch definierbar: $first(x),\ last(x),\ X\subseteq Y$, uvm.
            \end{itemize}
    \subsection{Definition}
        \begin{itemize}
            \item aromare Formeln sind bei und Prädikate $R_i^j\subseteq D^j$
            \item $D$ ist Domäne, also Menge aller Wortpositionen
            \item $R_i^j(x_1,\dots,x_j)$ ist atomare Formel (mit freien Variablen)
            \item Sind $\varphi_1,\varphi_2$ Formeln, so auch $\varphi_1\wedge\varphi_2$ und $\neg \varphi_1$
            \item Ist $\varphi$ Formel, so auch $\exists x\varphi$ und $\exists X\varphi$
        \end{itemize}
    \subsection{Notation}
        Die Notation ist abhängig davon, welche ``Bausteine'' eine Formel verwendet. Beispielsweise:
        $FO[<]$: Quantifizierung erster Ordnung und $<$-Prädikat ($Q_a$ Quantor immer gegeben)\\
        \subsubsection{Anmerkungen}
            \begin{itemize}
                \item $FO[+1]\subsetneq FO[<]$
                \item $MSO[+1]=MSO[<]=MSO$ ``monadic second order''
                \item $FO+MOD[<]$: $MOD_k:$ Durch $k$ teilbare Zahl (von `a's) $MOD_kxQ_ax$
                \item $FO[+]:$ $x+y=z$
            \end{itemize}
