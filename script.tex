\documentclass[german,a4paper,10pt]{scrreprt}
\usepackage{fancyhdr}
\usepackage[ngerman]{babel}
\usepackage{graphicx}
\usepackage{dsfont}
\usepackage{amsthm}
\usepackage{amsmath,amssymb}
\usepackage{stmaryrd}
\usepackage{xfrac}
\usepackage[utf8]{inputenc}
\usepackage{enumerate}
\usepackage{hyperref}
\usepackage{tikz}
\usepackage{cancel}
\usetikzlibrary{positioning,shadows,arrows,automata,shapes,decorations.pathreplacing}


\newcommand{\qq}[1]{``#1''}
\newcommand{\dcup}[0]{\ensuremath{\mathbin{\dot{\cup}}}}
\newcommand{\Sigmah}[0]{\ensuremath{\hat{\Sigma}}}
\newcommand{\Sigmapush}[0]{\ensuremath{\Sigma_{push}}}
\newcommand{\Sigmaneut}[0]{\ensuremath{\Sigma_{neutral}}}
\newcommand{\Sigmapop}[0]{\ensuremath{\Sigma_{pop}}}
\newcommand{\Tau}{\mathcal{T}}

\title{Formale Sprachen II}
\author{Vorlesung: Michael Ludwig, Sebastian Schöner\\
Mitschrieb: Jan-Peter Hohloch}
\date{WS 15/16}

\begin{document}
    \maketitle
    \tableofcontents
    \newpage
    \chapter{Einführung}
        %Do not touch first page!
\section{Formale Sprachen}
    \begin{itemize}
        \item Was ist die Theorie Formaler Sprachen?
        \begin{itemize}
            \item schlecht definierter Begriff!
        \end{itemize}
        \item Sei $\Sigma$ eine \emph{endliche} Menge, die wir Alphabet nennen
        \item $\Sigma^*$ ist die Menge der \emph{endlichen} Sequenzen von $\Sigma$-Elementen
        \item Eine (formale) Sprache ist die Teilmenge von $\Sigma^*$
        \item Es gibt $2^{|\Sigma^*|}>|\mathds{N}|$ Sprachen, uns interessieren nur abzählbar viele davon.
        \item Wir stellen Sprachen durch endliche Beschreibungen dar
            \begin{itemize}
                \item Grammatik
                \item Automaten (Endlich, Keller, Turingmaschine,\dots)
                \item Logik
                \item Algebra
                \item Schaltkreise
                \item Ausdrücke
                \item \dots
            \end{itemize}
    \end{itemize}\vspace*{-2cm}\hspace*{6cm}\vspace*{-7cm} %!!!
    \begin{tikzpicture}[node distance=4cm,->]
        \node[rectangle,align=center,draw,text width=2cm] (TM) {Turing\-maschine};
        \node[rectangle,align=center,draw,text width=3cm] (BT) [above of=TM,yshift=-2cm] {Berechenbarkeits-\\theorie};
        \node[rectangle,align=center,draw,text width=2cm] (TFS) [below left of=TM] {Theorie Formaler\\ Sprachen (\qq{leicht})};
        \node[rectangle,draw,text width=2cm] (K) [below right of=TM] {Komplexitäts\-theorie \\(\qq{schwer})};
        \node (l) [below right of=TFS,align=center,text width=2cm] {geringe Komplexität};
        \path (TM) edge node[right,align=center,text width=2cm]{mächtiger} (BT)
            (TM) edge node[xshift=.3cm,right,align=center,text width=2cm]{Platz/Zeit\\beschränken} (K)
            (TM) edge node[xshift=-.3cm,left,align=center,text width=2cm]{\qq{Funktionalität} einschränken} (TFS)
            (TFS) edge (l)
            (K) edge (l);
    \end{tikzpicture}\newpage
\section{Wiederholung Theoretische Informatik}
    \subsection{Chomsky-Hierarchie}
        \begin{itemize}
            \item[] Alle Sprachen
                \begin{itemize}
                    \item[$\supset$] Typ 0 (Turingmaschine)
                    \begin{itemize}
                        \item[$\supset$] Typ 1 (kontextsensitiv) ($\mathcal{O}(n)$ Platz)
                        \begin{itemize}
                            \item[$\supset$] Typ 2 (kontextfrei) (Kellerautomat)
                            \subitem$\supset$ Typ 3 (regulär) (endlicher Automat)
                        \end{itemize}
                    \end{itemize}
                \end{itemize}
        \end{itemize}
    \subsection{Berechenbarkeit}
        \begin{itemize}
            \item Church-Turing-These
            \begin{itemize}
                \item Sprache entscheidbar $\Leftrightarrow$ charakteristische Funktion berechenbar ($\Rightarrow$ Wortproblem, andere lassen sich darauf abbilden)
            \end{itemize}
            \item Reduktion: $A\le_m B :\Leftrightarrow \left(\exists f: x\in A\Leftrightarrow f(x)\in B\right)$ (many-one Reduktion)
            \item Reduktion (Turing): $A\le_T B:\Leftrightarrow\exists TM$ mit Orakel $B$, die $A$ erkennt.
        \end{itemize}
    \subsection{Komplexitätstheorie}
        \begin{itemize}
            \item L ist C-schwer $:\Leftrightarrow\forall X\in C: X\le L$
            \item L ist C-vollständig $:\Leftrightarrow$ L ist C-schwer und L$\in$ C
        \end{itemize}
        \begin{figure}
            \centering
            \label{fig:complexity}
            \begin{tikzpicture}
                [node distance=0.1\textheight]
                \node (PSP) {PSPACE};
                \node (NP) [below left of=PSP] {NP};
                \node (NSP) [below right of=PSP] {NSPACE(n)=Typ 1};
                \node (P) [below of=NP] {P};
                \node (NL) [below right of=P] {NL$\supseteq$ Typ 2};
                \node (DSP) [below of=NSP] {DSPACE(n)};
                \node (L) [below of=NL] {L};
                \node (NC) [below of=L] {NC$^1\supseteq$ Typ 3};
                \node (TC) [below of=NC] {TC$^0$};
                \node (ACC) [below of=TC] {ACC$^0$};
                \node (AC) [below of=ACC] {AC$^0$};
                \path (PSP) edge (NP)
                    (PSP) edge (NSP)
                    (NP) edge (P)
                    (NSP) edge (DSP)
                    (P) edge (NL)
                    (DSP) edge (NL)
                    (NL) edge (L)
                    (L) edge (NC)
                    (NC) edge (TC)
                    (TC) edge (ACC)
                    (ACC) edge (AC);
                \draw[decorate, decoration={brace, amplitude=10pt}] (3,.2) -- node[right,xshift=10pt]{TM} (3,-7.4) ;
                \draw[decorate, decoration={brace, amplitude=10pt}] (3,-7.4) -- node[right,xshift=10pt]{Schaltkreis} (3,-14.3) ;
                \draw[dashed] (-5,-7.4) -- (5,-7.4);
            \end{tikzpicture}
            \caption{Komplexitätsklassen}
        \end{figure}
    \subsection{Theorie formaler Sprachen}
        \subsubsection{Reguläre Sprachen} % (fold)
        \label{ssub:intro-reg}
            \begin{itemize}
                \item REG = Typ3-Grammatik = DEA = NEA = reguläre Ausdrücke = endliche Monoide
                \item NEA $\rightarrow$ DEA: Potenzmengenkostruktion
                \item DEA: Minimalautomat, vgl. Myhill-Nerode
                \item Abschlüsse: Komplement, Schnitt, Vereinigung, Konkatenation, Stern, Homomorphismen, inverse Homomorphismen, Quotienten, \dots
                \item Entscheidbarkeit: Wort, Leerheit, Äquivalenz, Endlichkeit
                \item Tradeoff: Abschluss und Entscheidbarkeit vs. Ausdrucksstärke
                \item Pumping Lemma
            \end{itemize}
        % subsubsection intro-reg (end)
        \subsubsection{Kontextfreie Sprachen} % (fold)
        \label{ssub:intro-kf}
            \begin{itemize}
                \item Kellerautomaten = Typ 2-Grammatiken
                \item Wortproblem: CYK $\mathcal{O}(n^3)$ Laufzeit
                \item CNF: $A\rightarrow BC,\ A\rightarrow a$ ($\Rightarrow$ binärer Ableitungsbaum)
                \item Abschlüsse: Vereinigung, Konkatenation, Stern
                \item nicht abgeschlossen unter: Komplement, Schnitt
                \item Entscheidbarkeit: Wortproblem, Leerheit
                \item nicht entscheidbar: Äquivalenzproblem
                \item Pumping Lemma
            \end{itemize}
        % subsubsection intro-kf (end)
\section{Wiederholung Formale Sprachen I}
    \subsection{Algebra} % (fold)
    \label{sub:intro-algebra}
        \begin{itemize}
            \item Homomorphismus: $\varphi(w_1w_2)=\varphi(w_1)\varphi(w_2)$
            \item Monoid: abgeschlossen, assoziativ und neutrales Element existiert
            \item $\varphi : \Sigma^* \rightarrow M,\ X\subseteq M$, dann ist $\varphi^{-1}(X)$ eine Sprache
            \subitem oder: $L\subseteq \Sigma^*$ wird von M erkannt $\Leftrightarrow L=\varphi^{-1}\left(\varphi\left(L\right)\right)$
            \item L regulär $\Leftrightarrow$ L wird von endlichem Monoid erkannt.
        \end{itemize}
    % subsection intro-algebra (end)
    \subsection{Logik} % (fold)
    \label{sub:intro-logik}
        \begin{itemize}
            \item Beispiel:
            \begin{itemize}
                \item $\exists x\forall y: y\le x\wedge Q_a(x) \in FO[\le]$ (\qq{Wort endet auf a})
                \item $\exists X$\dots (monadische Logik 2. Ordnung (MSO))
            \end{itemize}
            \item später mehr
        \end{itemize}
    % subsection intro-logik (end)

    \chapter{Visibly Pushdown Languages (VPL)}
        Alur, Madhusudan 2004 / Mehlhorn 1980 (input driven PDA)
\section{Motivation}
    \begin{itemize}
        \item Tradeoff: Ausdrucksstärke vs. Abschlusseigenschaften, Entscheidbarkeit
        \item guter Tradeoff: VPL
        \item PDA (Kellerautomat): Akzeptiert per leerem Keller
        \item DPDA (deterministischer Kellerautomat): Akzeptiert per Endzustand
        \item VPA ist eingeschränkter DPDA:
        \begin{itemize}
            \item Eingabezeichen bestimmen Kelleraktion
            \item Partitionierung des Alphabets:
            \subitem $\Sigma=\Sigmapush\dcup\Sigmapop\dcup\Sigmaneut$
            \subitem $\Sigmah=\left(Sigma_{push},\Sigmapop,\Sigmaneut\right)$
            \subitem push = call, pop = return, neutral = internal
        \end{itemize}
        \item Beispiel: $\Sigmah=\left(\{a\},\{b\},\{c\}\right)$
        \begin{itemize}
            \item $a^nb^n$
            \item $(a^nb^n)^*$
            \item $\{w\mid|w|_a=|w|_b\}\not\in VPL$
            \subitem $\rightarrow$ bei negativem Keller nie in VPL
        \end{itemize}
        \item Denkmodell: $\Delta w=|w|_{\Sigmapush}-|w|_{\Sigmapop}$
        \subitem $\rightarrow$ hat \qq{Zacken}, muss am Ende des Wortes wieder bei 0 sein
    \end{itemize}
\section{Definition: PDA}
    $M=(Q,Q_S,Q_F,\Gamma,\Sigmah,\delta,\#)$ mit:
    \begin{itemize}
        \item $Q$: Zustände
        \item $Q_S$: Startzustände
        \item $Q_F$: Endzustände
        \item $\Gamma$: Kelleralphabet
        \item $\hat\Sigma$: Eingabealphabet
        \item $\#$: Kellerbodenzeichen
        \item $\delta$: Übergänge
        \subitem $\delta \subseteq Q\times \Sigmaneut \times Q \\\cup Q\times \Sigmapush \times \Gamma\setminus\{\#\}\times Q\\\cup Q\times \Sigmapop \times\Gamma\times Q$
    \end{itemize}
\section{Semantik}
    \begin{itemize}
        \item Konfiguration: Element $Q\times\Sigmah^*\times \Gamma^*$
        \item Konfigurationsübergänge: Relation $\rightarrow \subseteq K\times K$
        \item $c\in\Sigmaneut:\ q,w_1,\dots,w_n,\gamma_1,\dots,\gamma_k\rightarrow \delta(q,c),w_1,\dots,w_{n-1},\gamma_1,\dots,\gamma_k$
        \item $a\in\Sigmapush:\ q,w_1,\dots,w_n,\gamma_1,\dots,\gamma_k\rightarrow q',w_1,\dots,w_{n-1},\gamma_1,\dots,\gamma_k,\gamma$
        \subitem $(q',\gamma)\in\delta(q,a)$
        \item $b\in\Sigmapop:\ q,w_1,\dots,w_n,\gamma_1,\dots,\gamma_k\rightarrow q',w_1,\dots,w_{n-1},\gamma_1,\dots,\gamma_{k-1}$
        \subitem $q'\in\delta(q,b,\gamma_k)$
        \item $L(M):=\left\{w\mid\exists s\in Q_S\exists f\in Q_F: s,w,\# \rightarrow^* f,\varepsilon,\# \text{ oder } \in\Gamma^*\#\right\}$
    \end{itemize}
\section{Alur-Kongruenz}
\subsection{Wiederholung: Myhill-Nerode}
    Myhill-Nerode-Relation in Abhängigkeit von $L\subseteq \Sigma^*$\\
    $u\sim_L v :\Leftrightarrow \forall x,y: xuy\in L \Leftrightarrow xvy\in L$ (syntaktische Kongruenz)\\
    $\Sigma^*/\sim_L=\{[w]|w\in\Sigma^*\}$ heißt syntaktisches Monoid
    \subsubsection{Anmerkungen}
        \begin{itemize}
            \item $[w_1][w_2]=[w_1w_2]$
            \item L regulär $\Leftrightarrow |\Sigma^*/\sim_L|$ endlich
        \end{itemize}
\subsection{Definition}
    $u,v\in WM$ (WM: Well-matched $\rightarrow$ akzeptieren mit leerem Keller im Endzustand)\\
    $u\simeq v :\Leftrightarrow \forall x,y\in\Sigma^*: xuy\in L\Leftrightarrow xvy\in L$
\subsection{Satz}
    $L$ ist VPL $\Leftrightarrow |WM/\simeq |$ endlich
\subsection{Anmerkungen}
    \begin{itemize}
        \item Operationen
            \begin{itemize}
                \item Konkatenation
                \item Extend:
                \subitem $[w]\rightarrow [awb],\ a\in\Sigmapush,b\in\Sigmapop$
            \end{itemize}
        \item $WM:=\{w\in\Sigma^* \mid \Delta(w)=0\wedge \forall i\le|w|:\Delta(w_1\dots w_i)\geq 0\}$
        \subitem mit $\Delta(w)=|w|_{\Sigmapush}-|w|_{\Sigmapop}$
        \item $\Sigma^* \supseteq L$ regulär $\rightarrow \Sigmah=(\emptyset,\emptyset,\Sigma)$
        \item Jedes visibly Wort lässt sich durch concat und extend erzeugen
    \end{itemize}
    \subsubsection{Beispiel}
    \vspace*{2cm}Leite $aaabbabb$ ab:\vspace*{-2.5cm}\\
        \hspace*{5cm}\begin{tikzpicture}
            \node (A) {$[\varepsilon]$};
            \node (Z) [right of=A] {$[\varepsilon]$};
            \node (B) [below of=A] {$[ab]$};
            \node (Y) [below of=Z] {$[ab]$};
            \node (C) [below of=B] {$[aabb]$};
            \node (D) [below right of=C] {$[aabbab]$}; \node[above of=D, yshift=-.5cm,xshift=-.1cm]{$\circ$};
            \node (E) [below of=D] {$[aaabbabb]$};
            \path (A) edge (B)
                (B) edge (C)
                (C) edge (D)
                (D) edge (E)
                (Z) edge (Y)
                (Y) edge (D);
        \end{tikzpicture}
\section{Satz}
    Für jeden VPA M existiert DVPA $M'$ mit L(M)=L($M'$) wobei L(M)$\subseteq$WM
    \subsubsection{Beweis}
    geg. $M=(Q,Q_S,Q_F,\Gamma,\Sigmah,\delta,\#)$, konstruiere $M'=(Q',q_S',Q_F',\Gamma',\Sigmah,\delta',\#)$
    \begin{itemize}
        \item $Q'=\mathcal{P}(Q\times Q)$
        \item $q_S'=\{(q,q)\mid q\in Q\}\in Q'$ (=$id_Q$)
        \item $Q_F'=\{S\in Q'\mid \exists s\in Q_S\exists f\in Q_F: (s,f)\in S\}$
        \item $\Gamma'=\Sigmapush\times Q'\cup \{\#\}$
        \item $\delta'$:
        \begin{itemize}
            \item $a\in\Sigmapush: \delta'(S,a)=(id_Q,(a,S))$
            \item $b\in\Sigmapop : \delta'(S,b,(a,S))=S''$\\%TODO: Formatting
            $S''=\{(q,q')\mid\exists q_1,q_2,q_3: (q,q_1)\in S', \exists\gamma\in\Gamma:(q_2,\gamma)\in\delta(q_1,a), (q_2,q_3)\in S, q'\in\delta(q_3,b,\gamma))\}$
            \item $c\in\Sigmaneut: \delta'(S,c)=\{(q,q')\mid \exists q''\in Q: (q,q'')\in S\wedge q'\in \delta(q'',c)\}$
        \end{itemize}
    \end{itemize}
\section{Satz von Chomsky-Schützenberger, 1963}
    Jede kontextfreie Sprache ist das homomorphe Bild einer Sprache $\mathds{D}_{2k}\cap R$ wobei $R$ regulär.\\
    $\forall L\subseteq \Sigma^*\text{ kf. }\exists k\in\mathds{N}, R\text{ reg. }\exists\varphi:L=\varphi(\mathds{D}_{2k}\cap R)$ $\varphi:\Sigma^*\rightarrow \{[_1,]_1,[_2,]_2,\dots\}^*$
    \subsection{Beweis}
        $L$ kf $\Rightarrow\exists G=(V,\Sigma,P,s)$ mit $L(G)=L$ und $L$ in CNF\\
        $X\in L(G)\Leftrightarrow\exists$ Ableitungsbaum (mit CNF: Binärbaum)\\
        Idee: Nutze Korrespondenz zwischen Bäumen und Dyck-Sprachen.\\\vspace{2mm}
        Konstruiere Grammatik $G'=(V',\Sigma',P',S')$ wobei $L(G')=\mathds{D}_{2k}\cap R$\\
        Sei $P=\{\Pi_1,\dots,\Pi_k\}$, dann $P'=\{\Pi_1',\dots,\Pi_k'\}$
        \begin{itemize}
            \item ist $\Pi_i=A\rightarrow BC$, dann $\Pi_i'=A\rightarrow\mathop{[}\limits_i^1B\mathop{]}\limits_i^1\mathop{[}\limits_i^2C\mathop{]}\limits_i^2$
            \item ist $\Pi_i=A\rightarrow a$, dann $\Pi_i'=A\rightarrow\mathop{[}\limits_i^1\mathop{]}\limits_i^1\mathop{[}\limits_i^2\mathop{]}\limits_i^2$
        \end{itemize}
        $\Sigma'=\{\mathop{[}\limits_i^1,\mathop{]}\limits_i^1,\mathop{[}\limits_i^2,\mathop{]}\limits_i^2,\dots,\mathop{[}\limits_k^1,\mathop{]}\limits_k^1,\mathop{[}\limits_k^2,\mathop{]}\limits_k^2\}$\\
        Wähle $\varphi:\Sigma'\rightarrow\Sigma$:
        \begin{itemize}
            \item Ist $\Pi_i=A\rightarrow BC$, so $\varphi(\mathop{[}\limits_i^1)=\varphi(\mathop{]}\limits_i^1)=\varphi(\mathop{[}\limits_i^2)=\varphi(\mathop{]}\limits_i^2)=\varepsilon$
            \item Ist $\Pi_i=A\rightarrow a$, so $\varphi(\mathop{]}\limits_i^1)=\varphi(\mathop{[}\limits_i^2)=\varphi(\mathop{]}\limits_i^2)=\varepsilon,\varphi(\mathop{[}\limits_i^1)=a$
        \end{itemize}
        \begin{enumerate}[1)]
            \item $L=\varphi(L(G'))$
            \item $L(G')=\mathds{D}_{2k}\cap R$
        \end{enumerate}
        \begin{enumerate}[1)]
            \item $x\in L \Leftrightarrow x\in L(G)\\\Leftrightarrow \exists$ Ableitungsbaum von G für $x\\\Leftrightarrow x'\in L(G'),$ wobei $x'$ Klammerrepräsentation des Baumes ist$\\\Leftrightarrow x\in\varphi(L(G'))$, da für jedes Blatt das richtige Zeichen übrig bleibt
            \item offensichtlich: $L(G')\subseteq\mathds{D}_{2k}$\\
            Weitere Bedingungen an die Wörter, um Konstruktion zu genügen:
            \begin{enumerate}[a)]
                \item einem $\mathop{]}\limits_i^1$ muss immer $\mathop{[}\limits_i^2$ folgen
                \item einem $\mathop{]}\limits_i^2$ muss $\mathop{]}\limits_j^1$ oder $\mathop{]}\limits_j^2$ folgen (beachte Sonderfall für Startvariable)
                \item ist $\Pi_i=A\rightarrow BC$, dann muss $\mathop{[}\limits_i^1$ ein $\mathop{[}\limits_p^1$ folgen mit $\Pi_p=B\rightarrow EF$ oder $\Pi_p=B\rightarrow b$ und $\mathop{[}\limits_i^2$ muss immer $\mathop{[}\limits_q^1$ folgen mit $\Pi_q=C\rightarrow GH$ oder $\Pi_q=C\rightarrow c$
                \item ist $\Pi_i=A\rightarrow a$, dann muss auf $\mathop{[}\limits_i^1$ $\mathop{]}\limits_i^1$ folgen und auf $\mathop{[}\limits_i^2$ $\mathop{]}\limits_i^2$ und auf $\mathop{]}\limits_i^1$ $\mathop{[}\limits_i^2$
                \item $\mathop{[}\limits_i^1$ als erstes Zeichen, dann muss $\Pi_i$ S auf der linken Seite haben
            \end{enumerate}
            \item[$\rightarrow$] Bedingungen sind alle regulär. Schneide alle Sprachen dieser Bedingungen; erhalte $R$, regulär.
        \end{enumerate}
\section{Satz von Greibach}
    \qq{Die schwerste kontextfreie Sprache}\\
    \subsection{Satz (Greibach, 1974)}
        Es gibt eine feste Sprache $L_0$, sodass $L$ kontextfrei $\Leftrightarrow \exists$ Homomorphismus $\varphi: L=\varphi^{-1}(L_0)$
        \subsubsection{Anmerkungen}
            \begin{itemize}
                \item Homomorphismen sind leicht zu berechnende Reduktionen
                \item Interpretation: Kann eine Maschine $L_0$ entscheiden, so auch jede andere kontextfreie Sprache
                \item $L$ kontextfrei $\Rightarrow\exists G:L=L(G),\ G$ in Greibachnormalform (GNF):
                \subitem $A\rightarrow aA_1A_2\dots A_k$
            \end{itemize}
        \subsubsection{Beweis}
            \emph{Idee:} Wähle $\varphi$ und $L_0$ so, dass sich Ableitung in GNF in $L_0$ wiederfindet.\\[0.2cm]
            $\Sigma=\{a_1,a_2,b_1,b_2,c,d,\$\},\ L_0\subseteq \Sigma^*,\ x_i,z_i\in\Sigma^*, y_1y_2\dots y_n\in\$\mathds{D}_2$\\
				$L_0 = \{\epsilon\}\{x_1cy_1cz_1d\dots x_0cy_ncz_nd| n\geq 1;x_i,z_i \in \Sigma^*; y_1,y_2,\dots,y_n\in \$\mathbb{D}_2$\\
            $L_0$ ist kontextfrei.\\[0,2cm]
            Sei G in GNF mit $L=L(G)$ und $G=\left(\{A_1,\dots,A_{|V|}\},\Gamma,P=\{\Pi_1,\dots,\Pi_k\},A_1\right),\ P_a=\{\Pi\in P\mid \Pi$ beginnt rechts mit $a\},\ \forall a\in\Gamma$\\
            $\varphi:\Gamma^*\rightarrow\Sigma^*, a\mapsto \left(\prod\limits_{\Pi\in P_a}c\Tau(\Pi)\right)cd=c\Tau(\Pi_{P_1})c\Tau(\Pi_{P_2})\dots c\Tau(\Pi_{|P_a|})cd$\\
            $\Pi_j=A_i\rightarrow aA_{j_1}\dots A_{j_m}$\\
            $\Tau(\Pi_j)=\underbrace{b_1b_2^ib_1}_{A_i}\underbrace{a_1a_2^{j_m}a_1}_{A_{j_m}}\dots \underbrace{a_1a_2^{j_1}a_1}_{A_{j_1}}$\\
            Sonderfall:\\
            $\Pi_j=a_1\rightarrow\dots \Rightarrow \Tau(\Pi_j)=\$a_1a_2^1a_1\Tau(\Pi_j)$ (wie oben)\\
            Anm: Startvariable kommt nur links vor
        \subsubsection{Beispiel}
            \begin{align*}
                \Pi_1=A_1\rightarrow aA_2A_2 && \Tau(\Pi_1)=\$a_1a_2^1a_1b_1b_2^1b_1a_1a_2^2a_1a_1a_2^2a_1\\
                \Pi_2=A_2\rightarrow a && \Tau(\Pi_2)=b_1b_2^1b_1\\
                \Pi_3=A_2\rightarrow bA_2 && \Tau(\Pi_3)=b_1b_2^1b_1a_1a_2^2a_2\\
                \Pi_4=A_2\rightarrow b && \Tau(\Pi_4)=b_1b_2^1b_1
            \end{align*}
            $\varphi(a) = c\Tau(\Pi_1)c\Tau(\Pi_2)cd\\
            \varphi(b) = c\Tau(\Pi_3)c\Tau(\Pi_4)cd\\
            A_1\Rightarrow_G^* aabA_2$\\
            \begin{align*}
                \varphi(aab) =& c\Tau(\Pi_1)c\Tau(\Pi_2)cdc\Tau(\Pi_1)c\Tau(\Pi_2)cdc\Tau(\Pi_3)c\Tau(\Pi_4)cd\\
                =& c\$a_1a_2^1a_1b_1b_2^1b_1a_1a_2^2a_1a_1a_2^2a_1cb_1b_2^2b_1cd\\
                &c\$a_1a_2^1a_1b_1b_2^1b_1a_1a_2^2a_1a_1a_2^2a_1cb_1b_2^2b_1cd\\
                &cb_1b_2^2b_1a_1a_2^2a_1cb_1b_2^2b_1cd
            \end{align*}
            Ableitung $aabA_2$:\\
            $$A_1\overset{\Pi_1}{\Rightarrow}aA_2A_2\overset{\Pi_2}{\Rightarrow}aaA_2\overset{\Pi_3}{\Rightarrow}aabA_2\dots$$
            $y_1y_2y_3$ \qq{kürzen}$\rightarrow \$a_1a_2^2a_1=\mu(y_1y_2y_3)$\\
            $y_4=\Tau(\Pi_4)\rightarrow y_1y_2y_3y_4\in\$\mathds{D}_2$
            Leite in Satzform immer die Variable am weitesten links ab.\\
            Allgemein:\\
            $(*) A_1'\rightarrow aB_1\dots B_m$\\
            $A_1\Rightarrow_G^* a_1\dots a_kA_1'\dots A_n \overset{(*)}{\Rightarrow} a_1\dots a_kaB_1\dots B_mA_2\dots A_n$\\
            $A_1\bar{A_1}\dots \mid A_n'\dots A_1'\leadsto A_n'\dots A_1'\bar{A_1'}B_m\dots B_1$
%TODO: 2015-11-10
	\\\\ wir zeigen:\\
	$S\Rightarrow_G^* a_1\dots a_kA_{i1}\dots A_{ir} \text{ mit Produktionen } (\Pi_{p1},\dots,\Pi_{pk})\\
	\Longleftrightarrow$\\
	\begin{enumerate}[(a)]
		\item $\varphi(a_1\dots a_k)=x_1c_{y1}c_{z1}d\dots x_kc_{yk}c_{zk}d$ \\\textit{$x_i$= nicht benutzte Produktionen ?}
		\item $y_1\dots y_k \in Prefix(\$\mathds{D}_2)$
		\item $\mu(y_1\dots y_k)= \$a_1a_2^{ir}a_1\dots a_1a_2^{i1}a_2$
	\end{enumerate}
	also $S\Rightarrow^* a_1\dots a_k \Leftrightarrow a) \cap y_1\dots y_n \in \$ \mathds{D}_2 \cap \mu(y_1\dots y_n)=\$$
\subsubsection{Beweis}durch Induktion über K\\
	
		IA:\\$ S\Rightarrow_G^1 a_1A_{i1}\dots A_{ir} \text{ dann }\\
		ex \Pi =S \rightarrow a_1A_{i1}\dots A_{ir} \in P_{a_1} \\
		\varphi(a_1)=x_1c_{y_1}c_{z_1}d= c\Tau(\Pi_{p1})c\Tau(\Pi_{p2})c\dots c \Tau(\Pi_{pm})d$\\\\
		$\Tau(\Pi)=\$a_1a_2a_1b_1b_2b_1a_1a_2^{ir}a_1\dots a_1a_2^{i1}a1$\\
		$	\rightarrow a) \checkmark b) \checkmark c)\checkmark$ offensichtlich.\\
		IS:  \\
		Induktions Annahme:\\
		$S\Rightarrow_G^*a_1\dots a_k A_{i1}\dots A_{ir} \Leftrightarrow a)\cap b) \cap c) \checkmark$\\
		sei $\Pi = A_{i1} \rightarrow a_{k+n} A_{j1}\dots A_{jt} \in P_a$ dann:\\
		$S \Rightarrow_G a_1\dots a_{k+1} % 1 oder n?
		A_{j1}\dots A_{jt}A_{i2}\dots A_{ir}$\\
		Induktiv:\\
		$\mu(y_1\dots y_k)=\$a_1a_2^{ir}a_1\dots a_1a_2^{i1}a_2\\
		\mu(y_1\dots y_k)\mu(y_{k+1})=\$a_1a_2^{ir}a_1\dots \cancel{a_1a_2^{i1}a_2} \cancel{b_1b_2^{i1}b_1}a_1a_2^{it}a_1\dots a_1a_2^{i1}a_1\\$
		a),b),c) \checkmark\\
		Rückrichtung: sind $\mu(y_1\dots y_k), \mu(y_1 \dots y_{k+1})$\\
		und $S\Rightarrow_G^* a_1\dots a_k A_{i1}\dots A_{ir}$ wie oben.\\
		So muss eine entgegengesetzte Regel exisitieren womit\\
		$S\Rightarrow_G^* a_1\dots a_{k+1}A_{j1}\dots A_{jt}A_{i2}\dots A_{ir}
		\hspace{2cm}\square$
		
		
	
    \section{Satz von Parikh}
    \newtheorem{def1}{Definition}[section]
	  \begin{def1}
	  	$M\subseteq W^n$ heißt linear g.d.w M ist der Form \\ $\{\alpha_0+n_1\alpha_1+\dots+n_m\alpha_m|n_i\geq0\}$ für $\alpha_1 \dots \alpha_m \in \mathds{N}^n$
	  \end{def1}
	  \begin{def1}
	  	M heißt semilinear g.d.w.
	  	$\exists$ lineare Mengen $M_1\dots M_k:$\\
	  	$M=\bigcup\limits_{i=1}^kM_i$
	  \end{def1}
	  \begin{def1}[Parikh Abbildung für Wörter] .\\
	  	$\Psi: \Sigma^*\rightarrow N^n$ wobei  $\Sigma=\{a_1\dots a_n\}$ und $w\mapsto (\underbrace{|w_{a_1}|}_{Anzahl a_1} ,|w_{a_2}|,\dots, |w_{a_n}|)$\\
	  	  Beispiel: $\phi(abbaccba)=\phi(aaabbbcc)=(3,3,2)$\\
	  \end{def1}
	\begin{def1}[Parikh Abbildung für Sprachen].\\
		$\Phi(L)=\{\Phi(w)|w\in L\}$\\
		$\Phi(xy)=\Phi(yx)=\Phi(x)+\Phi(y)$\\
		$\Phi(L)$ heißt Parikhbild von L
	\end{def1}
	\begin{def1}
		$L_1$ und $L_2$ heißen Zeichenäquivalent g.d.w $\Phi(L_1)=\Phi(L_2)$\\
	\end{def1}
	\subsubsection{Satz:}
	L hat semilineares Parikhbild $\Longleftrightarrow$ L ist zeichenäquivalent zu einer reg. Sprache
	\subsubsection{Beweis:}
	$\Phi(L) semilinear \Rightarrow \Phi(L)= M_1\cup M_2\cup M_m$ \\mit $M_i =\{\alpha_{i0}+n_1\alpha_{i1}+\dots n_{ir_{i}}\alpha_{ir_{i}}|n_{i,j}\geq0, 1\leq j \leq r_i\}$\\
	seien $y_{ij} \in \Sigma^*$ mit $\Phi(y_{ij})=\alpha_{ij}$\\
	Det Grammatik G: $s\rightarrow A_1|\dots|A_m$\\
	$A_i\rightarrow A_iy_{i1}| A_iy_{i2}|\dots| A_iy_{ir}|y_{i0}$\\
	es ist $\Phi(L)=\Phi(L(G))$ und $L(0)$ regulär.\\
	L $\Leftarrow$ Induktion über reguläre Ausdrücke.
	\begin{enumerate}
		\item $L=L(0) \Rightarrow \Phi(L)=\emptyset$
		\item $L=L(\epsilon) \Rightarrow \Phi(L)=0$
		\item $L=L(a_i) \Rightarrow \Phi(L)= \{0\dots,1,\dots 0\}$
		\item $L=L_1 \cup L_2, \Phi(L_1),\Phi(L_2) semilin \Rightarrow \Phi(L)= \Phi(L_1\cup L_2)=\Phi(L_1)\cup \Phi(L_2)$
		\item $L=L_1L_2 \hspace{0.2cm}\Phi(L_1),\Phi(L_2) s.l. \Rightarrow \Phi(L)= \Phi(L_1L_2)=\Phi(L_1)+ \Phi(L_2)$
		\item $\Phi(L) s.l. \Rightarrow \Phi(L^*) s.l$\\
		$\Phi(L)=M_1\cup\dots\cup M_k =\Phi(L_1)\cup\dots\cup\Phi(L_k)$\\
		$\Phi(L_i^*)$ s.l. da $\Phi(L_i^*)=\{0\}\cup\Phi(L_i)$\\ %versteh ich nicht...
		$\Phi(L_1^*L_2^*\dots L_k^*) s.l. = \Phi((L_1\cup L_2\dots \cup L_k)^*)=\Phi(L^*) \hspace{2cm} \square$
	\end{enumerate}
	
	\subsection{Satz (Parikh 1966)}
	L k.f $\Rightarrow$ L hat semilineares Prarikbild\\
	Ohne Beweis.\\
	
	
	Konsequenzen:\\
	\begin{itemize}
		\item jede kontextfreie Sprache. ist zeichenäquivalent zu einer regulären Sprache
		\item $L\subseteq\{a\}^*$ k.f. $\Rightarrow$ L reg 
	\end{itemize}

\section{Irgendwas fehlt hier immernoch}
    %TODO sectioning
    \dots\\
    Eindeutigkeit von $\hat{h}$:\\
    D.h. falls $g:\Sigma^*\rightarrow T^*$ Homomorphismus mit $g(a)=\hat{h}(a)\forall a$, dann ist $g(w)=\hat{h}(w)\forall w\in\Sigma^*$
    \subsubsection{Beweis}
        durch Induktion über die Wortlänge $|w|$:
        \begin{itemize}
            \item[$|w|=0$:] $g(w)=\epsilon=\hat{h}(w)$
            \item[$|w|=1$:] $g(a)=\hat{h}(a)\forall a$ nach Annahme
            \item[$|w|>1$:] Für beliebige aber feste Wortlänge $n$ gelte $g(w)=\hat{h}(w)$.\\
            \begin{math}
            \begin{array}{lrl}
                |w|=n+1:\ & w&=w'a \\
                & g(w'a) &= g(w')g(a)\\
                && \overset{IA}{=} \hat{h}(w')\hat{h(a)}\\
                && = \hat{h}(w'a)\\
                &&=\hat{h}(w)
            \end{array}
            \end{math}\\\hfill\qed
        \end{itemize}
    $L$ kontextfrei $\Rightarrow L=L(G),\ G$ in CNF:\\
    $A\rightarrow a\leadsto h(a)\\ A\rightarrow BC$
    \subsection{homomorphes Bild?}
    Induktion über Ableitungen\\
    \subsubsection{Inverser Homomorphismus}
    am Beispiel endlicher Automaten\\
    zu zeigen: Ist $L$ regulär, so auch $h^{-1}(L)$, $h: T^*\rightarrow \Sigma^*$\\
    Sei $A$ ein endlicher Automat für $L$, $A=(Q,\Sigma,\delta,q_0,F)$\\
    Baue Automat für $h^{-1}(L)$:\\
    Bei Eingabe $W$ berechne $h(w)$ buchstabenweise und simuliere $A$. $B=(Q,T,\delta',q_0,F)$ mit $\delta'(t,q)=\hat{\delta}(h(t),q),\ t\in T,\ q\in Q$.\\
    Funktioniert genauso für PDA und DPDA.
    \subsection{Anmerkungen}
    \begin{itemize}
        \item Homomorphismus kann mit $h(\$)=\epsilon$ $w\$ w^R$ kaputt machen
        \item nicht-löschender / $\epsilon$-freier Homomorphismus: $h(a)\not= \epsilon\ \forall a\in\Sigma$
        \item lässt sich trotzdem zerstören: $h(\$)=a,\ a\in\Sigma\setminus\{\$\}$
        \item $\mathds{D}_k$ sind VPL
        \item VPL sind abgeschlossen unter $\cap$
        \item reguläre Sprachen sind VPL
        \begin{itemize}
            \item Chomsky-Schützenberger
            \item PDA auf $\Sigma$ $\leadsto$ PDA auf $(\Sigma_c,\Sigma_r,\Sigma_{int})$
            \begin{itemize}
                \item $h(a_c)=h(a_r)=h(a_{int})=h(a)$
                \item $\Sigma_c=\{a_c\mid a\in\Sigma\}$
                \item $\Sigma_r=\{a_r\mid a\in\Sigma\}$
                \item $\Sigma_{int}=\{a_{int}\mid a\in\Sigma\}$
            \end{itemize}
            \item Kellerinhalte einer kontextfreien Sprache sind immer regulär.
        \end{itemize}
    \end{itemize}

    \chapter{Algebra}
        \section{Wiederholung}
    \subsection{Eigenschaften}
    Betrachte Menge $M$ und Verknüpfung $\circ: M\times M\rightarrow M$ ($\circ(m_1,m_2)=m_1\circ m_2$).\\
    $(M,\circ)$ oder kurz $M$ kann verschiedene Eigenschaften haben:\\
    \begin{tabular}{c|cccc}
     & assoziativ & neutrales Element & inverse Elemente & kommutativ\\\hline
     Halbgruppe & $\times$ & && ?\\
     Monoid & $\times$ & $\times$ & & ?\\
     Gruppe & $\times$ & $\times$ & $\times$ & ?\\
     abelsche Gruppe & $\times$ & $\times$ & $\times$ &$\times$
    \end{tabular}
        \subsubsection{Beispiel}
            \begin{itemize}
                \item $(\mathds{N},\cdot,1)$ Monoid
                \item $(\mathds{N},+,0)$ Monoid
                \item $(\mathds{Z},+,0)$ Gruppe
                \item $(2^X,\cap,X)$ kommutatives Monoid
                \item $(\Sigma^*,\cdot,\epsilon)$ Monoid
            \end{itemize}
    \subsection{Multiplikationstabelle}
    Sei $M$ endl. Halbgruppe: \begin{math}
        \begin{array}
            {c|ccc}
            & m_1 & m_2 & \dots\\\hline
            m_1 & m_1^2 & m_1m_2 &\\
            m_2 & m_2m_1 & m_2^2 &\\
            \vdots &
        \end{array}
    \end{math}
    leicht ablesbar:
    \begin{itemize}
        \item existiert neutrales Element?
        \item ex. 0-Element?
        \item kommutativ?
    \end{itemize}
    \subsubsection{Beispiel}
    $(\mathds{Z}/3\mathds{Z},+)\rightarrow (\mathds{Z}_3,+),\ +:\mathds{Z}^2\rightarrow \mathds{Z}_3:x+y\mapsto (x+y)\mod 3$\\
    \begin{math}
        \begin{array}
            {c|ccc}
            & \left[0\right] & \left[1\right] & \left[2\right]\\\hline
            \left[0\right] & 0 & 1 & 2\\
            \left[1\right] & 1 & 2 & 0\\
            \left[2\right] & 2 & 0 & 1
        \end{array}
    \end{math}
    \subsection{Homomorphismen}
        Sei $\varphi: G\rightarrow H$ und $(G,\cdot),(H,\circ)$ Halbgruppen.\\
        $\varphi$ ist Homomorphismus $\Leftrightarrow\forall x,y\in G: \varphi(x\cdot y)=\varphi(x)\circ\varphi(y)$ bei Monoiden: $\varphi(e_G)=e_H$\\
        $\varphi$ heißt:
        \begin{itemize}
            \item injektiv $\Leftrightarrow\forall h\in H:|\varphi^{-1}(h)|\le 1$ (Monomorphismus, $\varphi: G \hookrightarrow H$)
            \item surjektiv $\Leftrightarrow\forall h\in H:|\varphi^{-1}(h)|\ge 1$ (Epimorphismus, $\varphi: G\twoheadrightarrow H$)
            \item bijektiv, $\Leftrightarrow\forall h\in H:|\varphi^{-1}(h)|= 1$ (Isomorphismus, $\varphi: G \overset{\sim}{=} H$)
            \item $\varphi : G\rightarrow G$ (Endomorphismus)
            \item bijektiver Endomorphismus heißt Automorphismus
        \end{itemize}
    \subsection{Relationen}
        $\sim\subseteq X\times X$ ist Relation, $(x_1,x_2)\in\sim\Leftrightarrow : x_1\sim x_2$\\
        $\sim$ reflexiv, symmetrisch, transitiv: Äquivalenzrelation\\
        \indent$\Rightarrow \sim$ partitioniert $X$ in Äquivalenzklassen $X/\sim$\\
        Sei $(H,\circ)$ eine Halbgruppe und $\sim\subseteq H\times H$. $\sim$ heißt verträglich mit $\circ$, wenn $a\sim b\wedge c\sim d\Rightarrow ac\sim bd$. $\sim$ ist damit eine Kongruenzrelation $\Rightarrow (H/\sim,\circ_\sim)$ ist Halbgruppe.\\
        $\circ: H\times H\rightarrow H: a\circ b\mapsto c$\\
        $\circ_\sim: H/\sim\times H/\sim\rightarrow H: [a]_\sim\circ [b]_\sim\mapsto [c]_\sim$
        \subsubsection{Normalteiler}
            $N<G$: N ist Untergruppe von G\\
            $N\triangleleft G:\ $
            N ist Normalteiler von G, d.h. $\forall g\in G : gN=Ng\ (\Leftrightarrow \forall g\in G : gNg^{-1}=N$)\\
            Normalteiler und Kongruenzrelation stehen n 1-zu-1-Zusammenhang: Nebenklassen $\leftrightarrow$ Äquivalenzklassen; z.B. $\mathds{Z}/n\mathds{Z}$
    \subsection{Freies Monoid}
        (freie Gruppe, ...)\\
        Zwei äquivalente Definitionen:
        \begin{enumerate}
            \item $M$ heißt frei über $A\subseteq M\Leftrightarrow \forall m\in M: m$ ist eindeutiges Produkt von Elementen aus $A$ (z.B. $\Sigma\subseteq\Sigma^*$)
            \item $M$ heißt frei über $A\subseteq M\Leftrightarrow \exists g:A\hookrightarrow M$ und $f:A\rightarrow X$ beliebig ($X$ Monoid) $\Rightarrow \exists!\varphi:M\rightarrow X$
        \end{enumerate}
        \subsubsection{Beispiel}
            \begin{itemize}
                \item $(\Sigma^*,\cdot)$ ist frei über $\Sigma$
                \item $(\mathds{N},+)$ ist frei über $\{1\}$
                \item jede Gruppe/Monoid ist Faktorgruppe/-monoid einer freien Gruppe $\rightarrow$ frei heißt frei von Relationen
                \item $\left(\left(\Sigma\cup\bar{\Sigma}\right)^*,\cdot \right)/_{a\bar{a}=1}\cong F(\Sigma)$ (freie Gruppe über $\Sigma$)
                \item $F(\Sigma)/_{xy=yx}$ ist freie abelsche Gruppe
            \end{itemize}
\section{Spracherkennung durch Monoide}
    Sei $L\subseteq\Sigma^*,\ M$ Monoid und $\varphi:\Sigma^*\rightarrow M$ Homomorphismus. $L$ wird von $M$ erkannt (mit $\varphi$) $\Leftrightarrow \varphi^{-1}\left(\varphi(L)\right)=L$\\
    äquivalent: $\exists X\subseteq M: \varphi^{-1}(X)=L$
    \subsubsection{Beispiele}
        \begin{itemize}
            \item $L=\Sigmas$ wird von $\{1\}$ mit $\varphi:x\mapsto 1$ erkannt.
            \item $L=\{w\in\Sigma\mid |w|\equiv 0\pmod 2\}$ wird von $\mathds{Z}/2\mathds{Z}$ erkannt
            \begin{itemize}
                \item $\varphi(a)=[1]$, also $L=\varphi^{1}([0])$
                \item $\varphi(\epsilon)=[0]$
                \item das ginge mit jedem $\mathds{Z}/n\mathds{Z}$ mit $n$ gerade
            \end{itemize}
            \item $\Sigmas$ erkennt $L$ beliebig mit $\varphi=id$
            \begin{itemize}
                \item unendliche Monoide können ``zu viel''
                \item Konzept nur für endliche Monoide interessant
                \item[$\rightarrow$] Ziel: für Sprache $L$ kleinstes erkennendes Monoid finden
            \end{itemize}
        \end{itemize}
    \subsection{Das syntaktische Monoid}
        \subsubsection{Kongruenzrelation}
            Myhill-Nerode-Äquivalenz:
            \begin{itemize}
                \item $x\sim^L_L y\Leftrightarrow\forall z:xz\in L\Leftrightarrow yz\in L$
                \item $x\sim^R_L y\Leftrightarrow\forall z:zx\in L\Leftrightarrow zy\in L$
                \item[$\rightarrow$] $\sim^L_L,\ \sim^R_L$ sind keine Kongruenzrelation
            \end{itemize}
            für Kongruenzrelation muss gelten:
            $$u\sim v\wedge w\sim x\Rightarrow uw\sim vx$$
            \textbf{Beispiel:}
            \begin{itemize}
                \item $L=ac^+bc^+$
                \item Wähle $u=v=a,\ w=b,\ x=c$
                \item $uw=ab\not\sim vx=ac$
            \end{itemize}
        \subsubsection{Syntaktische Kongruenz}
            $x\sim_L y :\Leftrightarrow \forall u,v\in\Sigmas:uxv\in L\Leftrightarrow uyv\in L$ es ist $x\sim_L y\Leftrightarrow x\sim_L^L y\wedge x\sim_L^R y$
        \subsubsection{Syntaktisches Monoid}
            Sei $L\subseteq\Sigmas$, dann ist $Synt(L):=\Sigmas/\sim_L$ das syntaktische Monoid\\
            $\eta_L:\Sigmas\rightarrow Synt(L),\ w\mapsto[w]$ heißt syntaktischer Morphismus
            \begin{itemize}
                \item[\underline{Satz:}] $L$ reg $\Leftrightarrow Synt(L)$ endlich
                \item[\underline{Satz:}] $Synt(L)$ erkennt $L$ mittels $\eta_L$
                \item[\underline{Def.:}] $N\prec M :\Leftrightarrow \exists M'<M\exists\psi:M'\twoheadrightarrow N$. ``$N$ teilt $M$''
                \item[\underline{Satz:}] $M$ erkennt $L\Leftrightarrow Synt(L)\prec M$
            \end{itemize}
            \begin{enumerate}[1)]
                \item $\varphi:M\rightarrow M$ Endomorphismus und Epimorphismus $\Leftrightarrow \varphi$ Monomorphismus, falls $|M|<\infty$
                \item $Synt(L)$ erkennt $L$
                \item $\prec$ ist transitiv
                \item $Synt(L_1\cap L_2)\prec Synt(L_1)\times Synt(L_2)$
            \end{enumerate}

\end{document}
